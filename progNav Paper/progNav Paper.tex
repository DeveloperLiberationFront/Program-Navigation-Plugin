\documentclass[conference]{IEEEtran}
\usepackage{cite}


% correct bad hyphenation here
\hyphenation{op-tical net-works semi-conduc-tor}


\begin{document}
%
% paper title
% Titles are generally capitalized except for words such as a, an, and, as,
% at, but, by, for, in, nor, of, on, or, the, to and up, which are usually
% not capitalized unless they are the first or last word of the title.
% Linebreaks \\ can be used within to get better formatting as desired.
% Do not put math or special symbols in the title.
\title{Our Tool: Navigating Program Flow in the IDE}


% author names and affiliations
% use a multiple column layout for up to three different
% affiliations
\author{\IEEEauthorblockN{Chris Brown, Justin Smith, Tyler Albert, and Emerson Murphy-Hill}
\IEEEauthorblockA{Department of Computer Science\\
North Carolina State University\\
Raliegh, North Carolina 27606\\
Email: \{dcbrow10, jssmit11, tralber2\}@ncsu.edu, emerson@csc.ncsu.edu}
}

% make the title area
\maketitle

% As a general rule, do not put math, special symbols or citations
% in the abstract
\begin{abstract}
Program navigation is a critical task for software developers. 
Unfortunately, the current state-of-the-art tools do not adequately support developers in simultaneously navigating both control flow and data flow (i.e. program flow). 
To assist developers in effectively navigating program flow we designed and implemented a tool that leverages powerful program analysis techniques while maintaining low barriers to invocation.
Our tool enables developers to systematically navigate program flow upstream and downstream within the Eclipse IDE.
Based on a preliminary evaluation, our tool is awesome!  
\end{abstract}

% no keywords




% For peer review papers, you can put extra information on the cover
% page as needed:
% \ifCLASSOPTIONpeerreview
% \begin{center} \bfseries EDICS Category: 3-BBND \end{center}
% \fi
%
% For peerreview papers, this IEEEtran command inserts a page break and
% creates the second title. It will be ignored for other modes.
\IEEEpeerreviewmaketitle



\section{Introduction}
Modern software systems contain millions of lines of source code. 
As software grows in size and complexity, developers increasingly rely on tools to help them navigate the programs they create. 
Program navigation is a central task tied to many critical activities, including exploring new code bases, debugging, and assessing security vulnerabilities.


Code navigation is a critical task for software developers. 

Successful developers do not investigate source code line by line starting at the top of the file. Instead, they navigate the code's hierarchical semantic structures ~\cite{robillard2004investigate}. Methodical and structured
Developers ask questions about control flow and data flow~\cite{latoza2010hard, Smith2015}. 

Typically source code is presented to developers linearly. However, effective developers navigate   

\section{Design Principles}
Derived these design principles by evaluating the limitations of existing program navigation tools. 
\subsection{Powerful Program Analysis}
Program navigation tools should leverage powerful program analysis techniques. 

\subsection{Low Barriers to Invocation}

\subsection{Full Program Navigation}
Something that says the same tool allows users to analyze data flow \textit{across} methods upstream and downstream.

\subsection{Enables Systematic and Complete Evaluation}

\subsection{In Situ Results}  


\section{Related Work}
Summary of related work. Possibly a table evaluating existing tools on various design principles.


\section{[Name of Tool]}
We implemented 

\section{Preliminary Evaluation}
Same tasks as from: \cite{Smith2015}
Administered a post-test questionnaire. 

\section{Results}
Hopefully, participants are faster/more accurate with the tool. Rate positive experience on the questionnaire.

\section{Discussion}
\section{Limitations}



\section{Conclusion}
The conclusion goes here.




% conference papers do not normally have an appendix


% use section* for acknowledgment
\section*{Acknowledgment}


The authors would like to thank...





% trigger a \newpage just before the given reference
% number - used to balance the columns on the last page
% adjust value as needed - may need to be readjusted if
% the document is modified later
%\IEEEtriggeratref{8}
% The "triggered" command can be changed if desired:
%\IEEEtriggercmd{\enlargethispage{-5in}}

% references section

% can use a bibliography generated by BibTeX as a .bbl file
% BibTeX documentation can be easily obtained at:
% http://mirror.ctan.org/biblio/bibtex/contrib/doc/
% The IEEEtran BibTeX style support page is at:
% http://www.michaelshell.org/tex/ieeetran/bibtex/
\bibliographystyle{IEEEtran}
% argument is your BibTeX string definitions and bibliography database(s)
\bibliography{progNavPaper}




% that's all folks
\end{document}


