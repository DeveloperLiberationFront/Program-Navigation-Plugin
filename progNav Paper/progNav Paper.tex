\documentclass[conference]{IEEEtran}
\usepackage{cite}


% correct bad hyphenation here
\hyphenation{op-tical net-works semi-conduc-tor}


\begin{document}
%
% paper title
% Titles are generally capitalized except for words such as a, an, and, as,
% at, but, by, for, in, nor, of, on, or, the, to and up, which are usually
% not capitalized unless they are the first or last word of the title.
% Linebreaks \\ can be used within to get better formatting as desired.
% Do not put math or special symbols in the title.
\title{Our Tool: Navigating Program Flow in the IDE}


% author names and affiliations
% use a multiple column layout for up to three different
% affiliations
\author{\IEEEauthorblockN{Chris Brown, Justin Smith, Tyler Albert, and Emerson Murphy-Hill}
\IEEEauthorblockA{Department of Computer Science\\
North Carolina State University\\
Raleigh, North Carolina 27606\\
Email: \{dcbrow10, jssmit11, tralber2\}@ncsu.edu, emerson@csc.ncsu.edu}
}

% make the title area
\maketitle

% As a general rule, do not put math, special symbols or citations
% in the abstract
\begin{abstract}
Program navigation is a critical task for software developers. 
Unfortunately, the current state-of-the-art tools do not adequately support developers in simultaneously navigating both control flow and data flow (i.e. program flow). 
To assist developers in effectively navigating program flow we designed and implemented a tool that leverages powerful program analysis techniques while maintaining low barriers to invocation.
Our tool enables developers to systematically navigate program flow upstream and downstream within the Eclipse IDE.
Based on a preliminary evaluation, our tool is awesome!  
\end{abstract}

% no keywords




% For peer review papers, you can put extra information on the cover
% page as needed:
% \ifCLASSOPTIONpeerreview
% \begin{center} \bfseries EDICS Category: 3-BBND \end{center}
% \fi
%
% For peerreview papers, this IEEEtran command inserts a page break and
% creates the second title. It will be ignored for other modes.
\IEEEpeerreviewmaketitle



\section{Introduction}
Modern software systems contain millions of lines of source code. 
As software grows in size and complexity, developers increasingly rely on tools to help them navigate the programs they create. 
Program navigation is a central task tied to many critical activities, including exploring new code bases, debugging, and assessing security vulnerabilities. 

While navigating programs, developers ask questions about control flow and data flow throughout the program~\cite{latoza2010hard, Smith2015}. We will refer to these two concepts together as \textit{program flow}. Developers are interested in navigating program flow to trace how data is modified across multiple method invocations.

Integrated development environments (IDEs) present code linearly in the order methods are defined. However, successful developers do not navigate source code linearly (line by line starting at the top of the file). Instead, they methodically navigate the code's hierarchical semantic structures~\cite{robillard2004investigate}. To resovle this conflict and realize their ideal navigation strategies, developers rely on program navigation tools. 

% Probably more to come in here? What are some current tools and why are they limited.
% Discuss program visualization tools vs navigation tools.

In this work we will design, implement, and evaluate a program navigation tool.
To address the limitations of existing program navigation tools, our tool will embody five key design principles (Section \ref{DesignPrinciples}). We will implement our tool as a plugin to the Eclipse IDE. 

\section{Design Principles}
\label{DesignPrinciples}
In this section we describe the design principles that we used to shape our tool. We derived these design principles by evaluating the limitations of existing program navigation tools.
 
\subsection{Powerful Program Analysis}
By leveraging powerful program analysis techniques, navigation tools can provide more accurate information.
For example, by analyzing abstract syntax trees (ASTs) and call graphs, tools can make proper references to variables and methods. 
Simple textual analysis may lead to inaccurate results, especially when programs include inheritance and duplicate variable names.

\subsection{Low Barriers to Invocation}
Barriers to invocation may inhibit adoption. 
As developers may wish to navigate multiple program paths concurrently, repetitively invoking the tool may be cumbersome, especially if barriers are high. 

\subsection{Full Program Navigation}
Developers are not only interested in traversing programs' call graph, but also how data flows through the call graph.
To do so, developers must inspect the relationship between methods as well as the methods themselves.
Often the methods of interest span across multiple source files.
Furthermore, program navigation tools should support this traversal both upstream and downstream. 
That is, tools should highlight variable assignments and also subsequent variable uses. 

\subsection{Enables Systematic Evaluation}
Program navigation tools should help developers keep track of where they have been and where they are going. Especially while attempting to resolve complex defects, developers may want to thoroughly explore all program paths. Their navigation tools should help them keep track of their progress.

\subsection{In Situ Results}  
Switching between views in the IDE can cause disorientation [Cite]. As developers navigate through code, navigation tools should present their results in that context. 
When navigation tools present results outside the code, developers are burdened with the cognitive load of translating those results back to the code.

\section{Related Work}
Summary of related work, including a table evaluating existing tools on various design principles.
Spoiler alert, none of the tools satisfy all of the principles.


\section{[Name of Tool]}
We implemented our tool as a plugin to the Eclipse IDE [cite]. We chose Eclipse because of its popularity and extensibility. 
Eclipse is one of the most widely used open source IDEs for Java development and it provides many extension points for plugins. 

\begin{table}
\centering
\caption{Participant Demographics}
\begin{tabular}{|c|c|c|c|}
%\rowcolor{gray!50}
\hline
Participant & Industry & Java &\multicolumn{1}{c|}{Previous} \
%\rowcolor{gray!50}
& Experience & Experience & \multicolumn{1}{c|}{Eclipse Use} \
\hline
P1 & 9 & 5 & Yes \
\hline
P2 & 0 & 6 & Yes \
\hline
P3 & 3 & 2 & No\
\hline
P4 & 5 & 0 & No \
\hline
P5 & 12 & 10 & Yes \
\hline
P6 & 0 & 3.5 & Yes \
\hline
P7 & 1 & 9 & Yes \
\hline
P8 & 5.5 & 3.5 & Yes \
\hline
\end{tabular}
\label{table:participants}
\end{table}

\section{Preliminary Evaluation}
We plan to perform a preliminary evaluation of our tool with approximately five to ten developers.
Our goals are to get some initial feedback on the usability of our tool and evaluate whether our approach shows promise in enabling developers to efficiently navigate program flow. 

We will ask developers to perform two tasks that involve program flow navigation (Task 1 and Task 3 from \cite{Smith2015}).
We will measure their accuracy, speed, and administer a post-test usability questionnaire. 

http://citeseerx.ist.psu.edu/viewdoc/summary?doi=10.1.1.584.6610
Adapted form PSSUQ. 7 point Likert scale. Replace “This system” with “This tool”	
System Quality (the average of items 1-6) and Interface Quality (the average of items 13-16)

Ask questions based on applicable categories from Nielsen’s usability heuristics (1992 finding usability problems)
\section{Results}
Hopefully, participants are faster/more accurate with the tool. Rate positive experience on the questionnaire.

\section{Discussion}

\section{Limitations}

\section{Conclusion}

\section*{Acknowledgment}

The authors would like to thank...





% trigger a \newpage just before the given reference
% number - used to balance the columns on the last page
% adjust value as needed - may need to be readjusted if
% the document is modified later
%\IEEEtriggeratref{8}
% The "triggered" command can be changed if desired:
%\IEEEtriggercmd{\enlargethispage{-5in}}

% references section

% can use a bibliography generated by BibTeX as a .bbl file
% BibTeX documentation can be easily obtained at:
% http://mirror.ctan.org/biblio/bibtex/contrib/doc/
% The IEEEtran BibTeX style support page is at:
% http://www.michaelshell.org/tex/ieeetran/bibtex/
\bibliographystyle{IEEEtran}
% argument is your BibTeX string definitions and bibliography database(s)
\bibliography{progNavPaper}




% that's all folks
\end{document}


